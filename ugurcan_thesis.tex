\documentclass[12pt,oneandhalf,chaparabic,ceng,ms,eng,oneside,pntc]{gsufbe}
% for Computer Engineering use option ceng
% for Industrial Engineeering use option ie
% for Logistics and Finance Management use option lfm
% for Mathematics use option math
% for MSc use option ms
% for PhD use option phd
% Don't change other options due to instutional regulations. 
% You can delete next line If your thesis does not have an appendix
\usepackage{appendix}

% Use your latex packages here
%\usepackage{indentfirst}
\usepackage{graphicx}
\usepackage{amsmath}
\usepackage{amsfonts}
\usepackage{amssymb}
\usepackage{enumitem}
\usepackage{amsthm}
\usepackage{subcaption}
\usepackage{booktabs}
\usepackage{array}
\usepackage[round]{natbib}
\usepackage{har2nat}
\usepackage{algorithmic}
\usepackage[ruled,noline]{algorithm2e}
\usepackage{titlesec}
\usepackage[section]{placeins}
\usepackage{float}

% End of Latex Packages
%
% Any personal Latex definition, decleration, etc.
\makeatletter
\let\old@includegraphics\includegraphics
\renewcommand{\includegraphics}[2][,]{%
  \setbox9=\hbox{\old@includegraphics[#1]{#2}}%
  \ifdim\wd9>\textwidth
    \old@includegraphics[#1,width=\textwidth]{#2}%
  \else
    \old@includegraphics[#1]{#2}%
  \fi%
}
\makeatother

% End of personal stuff
%
% Personal Information
% ----------------------------
%
% Please check this part and fill in information about your thesis
%
% Name and Surname
\author{Uğurcan Ergün}
% Thesis Title English and Turkish
\title{Evaluating Feasibility of Container Virtualization for Network Function Virtualization}
\trtitle{Ağ İşlevi Sanallaştırma için Konteynır Sanallaştırmanın Uygunluğunun İncelenmesi}
% Department : English and Turkish
%
% The departments are pre-defined, you need not redeclare them. 
% Date : You should indicate the month of your thesis defence in English.
% Default is this month
%
\date{June 2018}
%
% Approval Page Details
% --------------------------
% For each command you can give the title as optional parameter enclosed in [ ]
%For white covered thesis, please comment out approval page in cls file.
%Committee members number:
%-------------------
%Single advisor/masters thesis: 2 members excluding advisor
%Two advisors/master thesis: 3 members excluding both advisors
%Single advisor/PhD Thesis: 4 members excluding advisor
%Two advisors/PhD Thesis: 5 members excluding advisor
%Please comment out extra member of committee. 
%
% prof : Prof. Dr.
% assocprof : Assoc. Prof. Dr.
% assistprof : Assist. Prof. Dr.
% dr : Dr.
%
% Supervisor
\supervisor[assistprof]{B. ATAY ÖZGÖVDE}
\departmentofsupervisor{Computer Engineering Department, GSU}
% co-Supervisor

%\cosupervisor[assocprof]{JANE DOE}
%\departmentofcosupervisor{Management, UCL}
% Ask your supervisor if you are not sure
\committeememberi[assistprof]{}
\affiliationi{}
\committeememberii[assistprof]{}
\affiliationii{}
%\committeememberiii[assistprof]{ATAY ÖZGÖVDE}
%\affiliationiii{Computer Engineering Department, GSU}
% Fourth committee member
%\committeememberiv[assocprof]{MAN DOE}
%\affiliationiv{Computer Engineering Department, MIT}
% Fifth committee member
%\committeememberv[assistprof]{WOMAN DOE}
%\affiliationv{Computer Engineering Department, TOBB ETÜ}
%
% Keywords : English & Turkish & French, Comma seperated No more than 5 keywords
\keywords{}
\motscles{}
\anahtarklm{}
%
% Abstract in English
%
\abstract{}
%
% Abstract in French
\resume{}
% Turkish Abstract
%
\oz{}
%
% Acknowledgements
\acknowledgments{}
%
% End of Personal and Introductory Information
%%%%%%%%%%%%%%%%%%%%%%%%%%%%%%%%%5
\setlength{\jot}{20pt}
%%% !!! This two should be last lines before \begin{document}, do no move them !!!
\usepackage[pdftex]{hyperref}
\usepackage[all]{hypcap}
\begin{document}
\addtolength{\textheight}{1.5cm}
% Preliminaries
\newlength\myindent
\setlength\myindent{6em}
\newcommand\bindent{%
  \begingroup
  \setlength{\itemindent}{\myindent}
  \addtolength{\algorithmicindent}{\myindent}
}
\newcommand\eindent{\endgroup}
\begin{preliminaries}
% If you are willing to use any custom stuff before Chapters, put it here
% Such as List of Abbreviations
% Check the abbreviations.tex for a template list of abbreviations
%\begin{theglossary}{TOSCA}
\item[API]   Application Programming Interface
\item[AUFS]  Advanced Multi-layered Unification File System
\item[BGP]   Border Gateway Protocol
\item[CAPEX] Capital Expenditures
\item[CDN]   Content Delivery Network
\item[CNI]   Container Network Interface
\item[CORD]  Central Office Re-architected as a Datacenter
\item[CPU]   Central Processing Unit
\item[CRI]   Container Runtime Interface
\item[ETSI]  European Telecommunications Standards Institute
\item[GiB]   Gigibyte
\item[HLC]   Home Location Register
\item[HSC]   Home Subscriber Server
\item[HTTP]  Hypertext Transfer Protocol
\item[JSON]  Javascript Object Notation
\item[I/O]   Input and Output
\item[IP]    Internet Protocol
\item[KVM]   Kernel Virtual Machine
\item[LXC]   Linux Containers
\item[MANO]  Management and Orchestration
\item[NAT]   Network Address Translation
\item[NFV]   Networking Function Virtualization
\item[OASIS] Organization for the Advancement of Structured Information Standards
\item[OPEX]  Operational Expenditures
\item[OS]    Operating System
\item[REST]  Representational State Transfer
\item[RNC]   Radio Network Controller
\item[SDN]   Software Defined Networking
\item[TOSCA] Topology and Orchestration Specification for Cloud Applications
\item[VNF]   Virtual Network Function
\item[VNFD]  Virtual Network Function Definition
\item[YAML]  Yet Another Markup Language
\end{theglossary}

% End of Preliminaries
\end{preliminaries}
%
% Latex content Goes Here
%
%
\newtheorem{thm}{Definition}[chapter]
\renewcommand{\thethm}{\arabic{chapter}.\arabic{thm}}
\newtheorem{prp}{Proposition}[chapter]
\renewcommand{\theprp}{\arabic{chapter}.\arabic{prp}}
\newenvironment{prf}{\noindent{\bf Proof}}{$\hfill \Box$ \vspace{10pt}}

\chapter{Introduction}

\chapter{Literature Review}

\chapter{Network Function Virtualization}
\section{Introduction} 
\section{ETSI MANO}
\subsection{Architecture}
\subsection{Implementations}
\section{Service Function Chaining}
\chapter{Container Virtualization}
\section{Introduction}
National Institute of Standards and Technology defines cloud computing as a model for enabling
easy access to a pool of computing resources that requires minimal intervention from the service
provider. With cloud services, any person or organization can buy computing resources from providers
without the need of investing in computing infrastructure themselves. 

One of the essential technologies for implementing cloud services is virtualization. In modern cloud
environments the usually preferred method for virtualization is called hypervisor based virtualization.
Xen and KVM can be called the two most common free and open source hypervisors. While Xen is used in
biggest public cloud platform Amazon Web Services, KVM is used in Google Cloud Platform and Openstack.

The main goal for the hypervisors is to isolate virtual machines in physical host to a degree that they
are no different than separate computers in a network. Virtual machines run a seperate operating system
called guest operating system and hypervisor mediates the system calls to the host operating system.
This technique is very efficient at isolating virtual machines but it also introduces an overhead
since there is an extra layer of abstraction. Previous studies show that while CPU and memory overhead
is minimal, biggest overheads occur in I/O operations. There is also some operational overheads that
comes from running an separate full-fledged guest OS. It also needs to be maintained and updated.

Recently there is another method of virtualization gaining traction in both research and the industry.
It's called operating system level virtualization or more commonly container virtualization.With
container virtualization virtual machines or containers have their own process and resource space but
they use host machines kernel and resources directly. Without the need of complete isolation and an
unaware guest OS container virtualization can achieve faster start up times, less resource consumption
and smaller images. But since all containers use the same kernel isolation between different containers
is weaker.

The earliest attempt for operating system level virtualization can be traced to the UNIX chmod command.
It is introduced in UNIX version 7 at year of 1979. By that time it could only provide file system
isolation. But FreeBSD jails is the first mechanism that can truly be called a container in a modern
sense. It was developed by Poul-Henning Kamp at 1998 and introduced in FreeBSD version 4 at 2000.
Jailed processed can't see other processes, can access only a specific part of their file systems and
have their own IP addresses. At Linux side, OpenVZ project developed a specialized kernel for running
containers at 2006. Contrary to jails, OpenVZ is more akin to a virtual machines. OpenVZ containers
have their own user structure, subject to configurable resource limits and can be migrated to separate
physical machines without the need of shutting down.

Despite both the concept and the technologies that make it possible were around for some time.
Containers are only recently started to see wide spread adoption. This can be explain with a couple
reasons. While Linux became one of the major platforms in the server domain there wasn't any container
technology available that can be run on an unmodified Linux kernel. Furthermore Linux kernel lacked
some of the essential capabilities to implement a container system such as resource management for
processes. Before cgroups was introduced in 2008 Linux kernel didn't have the capacity for controlling
and limiting the resources of a process or group of processes. Using cgroups and earlier introduced
kernel namespaces in the same year first native Linux
container project LXC was announced. While it didn't see much adopted by itself. It formed the basis
for the most of the modern container technologies.
\section{Docker}
Docker is a container virtualization technology that is developed in 2013 by a software company called
dotCloud. Among the modern container technologies Docker is the most widely used one and it would also
be fair to say Docker project's success is partially responsible for the recent adoption of container
virtualization technology. This might be attributed to Docker's different approach on how to prepare 
and use containers.

Virtual machines are used to be considered as virtual equivalents of physical dedicated servers and for
some time containers are also thought as faster and more lightweight virtual machines. But the main
contribution of Docker is focusing into an application instead of a server. Docker provides tools for
packaging an application and all of it’s dependencies together to run the application inside of a
container. Furthermore Docker proposes that each component of an application should be run in a 
different container thus encouraging developers to use a loosely coupled software architecture that is 
also called micro service architecture. According to micro service architectures managing and updating
individual components of applications are easier compared to managing a monolithic application which
all of it’s components runs in same environments and problems in one component can affect the whole
application.


\section{Kubernetes}

\chapter{If we have some future implementations}

\chapter{Experimentation}
\section{Setup}
\section{Experiment Design}
\section{Results}

\chapter{Conclusion}

\appendix
\thispagestyle{empty}
%\chapter[]{Proof of Some Theorem}
%\thispagestyle{empty}
%This is appendix text.

%\curriculumvitae
\label{chapter:vita}
Uğurcan Ergün was born on June 19, 1991 in Istanbul. He graduated from Istanbul Köy Hizmetleri
Anatolian High School at 2009. He studied Computer Engineering in Kocaeli University and got his
Bachelor's Degree at 2013. In September 2013 he became a software developer for the Istanbul based
cloud service provider Skyatlas Inc. Since October 2015 he is a graduate student at Galatasaray
University.

%\section*{\uppercase{Publications}}
%\begin{itemize}
%\item If you have publications you must write there.
%\end{itemize}
%\thispagestyle{empty}

\end{document} 